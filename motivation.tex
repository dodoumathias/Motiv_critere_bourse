\documentclass[12pt,a4paper]{article}

\usepackage[utf8]{inputenc}
\usepackage[T1]{fontenc}
\usepackage[french]{babel}
\usepackage{geometry}
\geometry{margin=2.5cm}

\usepackage{setspace}
\onehalfspacing

\begin{document}

\section*{Éléments essentiels d’une lettre de motivation pour une bourse}

Une lettre de motivation pour une bourse doit mettre en évidence plusieurs éléments clés permettant au comité de sélection d’évaluer la pertinence et l’impact du soutien financier accordé. Ces éléments peuvent être structurés autour de quatre piliers fondamentaux.

\subsection*{1. Parcours académique et mérite}

Le candidat doit présenter de manière claire et synthétique son parcours académique, en mettant en avant :
\begin{itemize}
    \item les diplômes obtenus ;
    \item le domaine d’études suivi ;
    \item les compétences clés acquises au cours de la formation ;
    \item le sérieux académique, les résultats obtenus ou l’engagement dans les études.
\end{itemize}

\textbf{Objectif :} démontrer que le candidat possède les capacités académiques nécessaires et qu’il mérite l’investissement consenti à travers la bourse.

\subsection*{2. Motivation et projet d’études}

Cette partie doit exposer la motivation profonde du candidat en répondant notamment aux questions suivantes :
\begin{itemize}
    \item pourquoi avoir choisi ce domaine d’études ;
    \item pourquoi poursuivre des études supérieures ou une spécialisation ;
    \item quelles connaissances ou compétences spécifiques le candidat souhaite acquérir.
\end{itemize}

\textbf{Objectif :} montrer que le candidat a une vision claire de son parcours académique et de ses objectifs d’apprentissage.

\subsection*{3. Projet professionnel et impact futur}

Le projet professionnel constitue un élément central de la lettre. Il doit préciser :
\begin{itemize}
    \item le métier ou le domaine d’activité envisagé à l’issue de la formation ;
    \item la manière dont les compétences acquises seront utilisées ;
    \item l’impact attendu pour la communauté, le pays ou le secteur concerné.
\end{itemize}

\textbf{Objectif :} démontrer que l’octroi de la bourse générera un impact positif et durable.

\subsection*{4. Situation personnelle et nécessité de la bourse}

Enfin, le candidat doit expliquer avec sobriété :
\begin{itemize}
    \item sa situation financière actuelle ;
    \item les raisons pour lesquelles la bourse est indispensable à la poursuite de ses études ;
    \item son engagement à réussir et à tirer pleinement profit de l’aide reçue.
\end{itemize}

\textbf{Objectif :} justifier le besoin réel de la bourse tout en soulignant le sérieux et la détermination du candidat.

\end{document}
